\input{configuration}

\def\ojoin{\setbox0=\hbox{$\bowtie$}%
  \rule[-.02ex]{.25em}{.4pt}\llap{\rule[\ht0]{.25em}{.4pt}}}
\def\leftouterjoin{\mathbin{\ojoin\mkern-5.8mu\bowtie}}

\title{Tutorial 6 --- Query Optimization, Planning, Evaluation }

\author{Richard Wong \\ \small \texttt{rk2wong@edu.uwaterloo.ca}}
\institute{Department of Electrical and Computer Engineering \\
  University of Waterloo}
\date{\today}


\begin{document}

\begin{frame}
  \titlepage

\end{frame}


\begin{frame}
\frametitle{Exercise 6-1}

Give instances of relations R and S that show that the following pairs of relational algebra expressions are not equivalent:

\begin{enumerate}
  \item $\pi_A(R - S)$ and $\pi_A(R) - \pi_A(S)$
  \item $\sigma_{\theta}(R \leftouterjoin S)$ and $R \leftouterjoin \sigma_{\theta}(S)$, where $\theta$ uses only attributes of $S$
\end{enumerate}

\end{frame}


\begin{frame}
\frametitle{Exercise 6-1 Solution (1/2)}

We are showing that $\pi_A(R - S)$ and $\pi_A(R) - \pi_A(S)$ are not equivalent. \\
Let our schemas be $R(A,B)$ and $S(A,B)$. \\
Let $R = \{(1, 2)\}$, $S = \{(1, 3)\}$.

\begin{align*}
  LHS &= \pi_A (R-S) \\
  &= \pi_A (\{(1, 2)\} - \{(1, 3)\}) \\
  &= \pi_A (\{(1, 2)\}) \\
  &= \{(1)\} \\
  RHS &= \pi_A (R) - \pi_A (S) \\
  &= \pi_A (\{(1, 2)\}) - \pi_A (\{(1, 3)\}) \\
  &= \{(1)\} - \{(1)\} \\
  &= \emptyset \\
  LHS & \neq RHS
\end{align*}
\end{frame}


\begin{frame}
\frametitle{Exercise 6-1 Solution (2/2)}

We are showing that $\sigma_{\theta}(R \leftouterjoin S)$ and $R \leftouterjoin \sigma_{\theta}(S)$ are not equivalent when $\theta$ uses only attributes of $S$. \\
Let our schemas be $R(A,B)$ and $S(A,C)$. \\
Let $R = \{(1, 2)\}$, $S = \{(42, 1337)\}$. \\
Let $\theta$ be a predicate like $C = 1$, or anything that satisifies no elements in C.

\begin{align*}
  LHS &= \sigma_{\theta}(R \leftouterjoin S) \\
  &= \sigma_{C=1}(\{(1, 2)\} \leftouterjoin \{(42, 1337)\}) \\
  &= \sigma_{C=1}(\emptyset) \\
  &= \emptyset \\
  RHS &= R \leftouterjoin \sigma_{\theta}(S) \\
  &= \{(1, 2)\} \leftouterjoin \sigma_{C=1}(\{(42, 1337)\}) \\
  &= \{(1, 2)\} \leftouterjoin \emptyset \\
  &= \{(1, 2, null)\} \\
  LHS & \neq RHS
\end{align*}
\end{frame}


\begin{frame}
\frametitle{Exercise 6-2}

Consider relations $R(A, B, C), S(C, D, E), T(E, F)$, where A, C, and E are their respective primary keys. \\

Suppose $n_R = 1000, n_S = 1500, n_T = 500$.


\begin{enumerate}
  \item What is the tightest upper bound we can place on $n_{ R \bowtie S \bowtie T }$?
  \item How could we compute the join efficiently?
\end{enumerate}
\end{frame}


\begin{frame}
\frametitle{Exercise 6-2 Solution}

$n_R = 1000, n_S = 1500, n_T = 500$. \\
Note that the size of the fully-joined relation ($n_{ R \bowtie S \bowtie T }$) will be the same no matter the order we execute the joins in, since natural joins are associative and commutative. \\
Suppose we consider the execution order $((R \bowtie S) \bowtie T)$.

\begin{align*}
  n_{ R \bowtie S } &\le n_R && \text{since C is a key of S} \\
  n_{ (R \bowtie S) \bowtie T } &\le n_{ R \bowtie S } && \text{since E is a key of T} \\
  n_{ R \bowtie S \bowtie T } &= n_{ (R \bowtie S) \bowtie T } && \text{by associativity} \\
  &\le n_{ R \bowtie S } \\
  &\le n_R \\
  &= 1000
\end{align*}

So $n_{ R \bowtie S \bowtie T } \le 1000$.

To efficiently compute the join, it helps to have indices on the primary keys of $S$ and $T$, and to use those indices to prevent linear scans of those relations during the join.

\end{frame}


\begin{frame}
\frametitle{Exercise 6-3}

Using the relational algebra equivalence rules, show how to derive the RHS expression from the LHS expression.

\begin{enumerate}
  \item $\sigma_{\theta_1 \land \theta_2 \land \theta_3}(R)$ = $\sigma_{\theta_1}(\sigma_{\theta_2}(\sigma_{\theta_3}(R)))$
  \item $\sigma_{\theta_1 \land \theta_2}(R \bowtie_{\theta_3} S)$ = $\sigma_{\theta_1}(R \bowtie_{\theta_3} \sigma_{\theta_2}(S))$, where $\theta_2$ uses only attributes of $S$
\end{enumerate}

\end{frame}


\begin{frame}
\frametitle{Exercise 6-3 Solution (1/2)}

\begin{align*}
  LHS &= \sigma_{\theta_1 \land \theta_2 \land \theta_3}(R) \\
  &= \sigma_{\theta_1 \land \theta_2}( \sigma_{\theta_3}(R) ) && \text{by $\sigma$-cascade (rule 1)} \\
  &= \sigma_{\theta_1}(\sigma_{\theta_2}(\sigma_{\theta_3}(R))) && \text{by $\sigma$-cascade (rule 1)} \\
  &= RHS
\end{align*}
\end{frame}


\begin{frame}
\frametitle{Exercise 6-3 Solution (2/2)}

\begin{align*}
  LHS &= \sigma_{\theta_1 \land \theta_2}(R \bowtie_{\theta_3} S) \\
  &= \sigma_{\theta_1}(\sigma_{\theta_2}(R \bowtie_{\theta_3} S) && \text{by $\sigma$-cascase (rule 1)} \\
  &= \sigma_{\theta_1}(R \bowtie_{\theta_3} \sigma_{\theta_2}(S)) && \text{since $\theta_2$ only uses attributes of $S$,} \\
  &= RHS && \text{we can distribute $\sigma_{\theta_2}$ over $\bowtie_{\theta_3}$}
\end{align*}
\end{frame}


\begin{frame}
\frametitle{Exercise 6-4}

Let $R$ be our relation with $n_r$ records. \\
Suppose $s_i$ records in $R$ match a predicate $\theta_i$: that is, $\sigma_{\theta_i}(R) = s_i$. \\
The \textit{selectivity} of $\theta_i$, $sel_{\theta_i}(R)$ is defined to be $\frac{s_i}{R}$. This represents the probability that a record in $R$ satisifies $\theta_i$.

Derive the selectivity formulas for the following complex selections:

\begin{enumerate}
  \item conjunction: $\sigma_{\theta_1 \land \theta_2 \land ... \land \theta_m}(R)$
  \item negation: $\sigma_{\lnot\theta}(R)$
  \item disjunction: $\sigma_{\theta_1 \lor \theta_2 \lor ... \lor \theta_m}(R)$
\end{enumerate}

\end{frame}


\begin{frame}
\frametitle{Exercise 6-4 Solution}

We make the simplifying assumption that predicates are independent of one another, allowing us to use standard probability rules for dealing with independent events.

\begin{enumerate}
  \item $sel_{\theta_1 \land \theta_2 \land ... \land \theta_m}(R) = \prod_{i=1}^{m} sel_{\theta_i}(R)$
  \item $sel_{\lnot\theta}(R) = 1 - sel_{\theta_i}(R)$
  \item $sel_{\theta_1 \lor \theta_2 \lor ... \lor \theta_m}(R) = 1 - sel_{\lnot \theta_1 \lor \theta_2 \lor ... \lor \theta_m}(R)$ \\
    $ = 1 - \prod_{i=1}^{m} sel_{\lnot \theta_i}(R)$ \\
    $ = 1 - \prod_{i=1}^{m} 1 - sel_{\theta_i}(R)$
\end{enumerate}

\end{frame}


\begin{frame}
\frametitle{Exercise 6-5}

What are some strategies that a query optimizer could use to reduce the cost of query plan selection, or the cost of the query itself?

\end{frame}


\begin{frame}
\frametitle{Exercise 6-5 Solution}

\begin{itemize}
  \item limit the size of intermediate results early on in the plan (select and project ASAP)
  \item cache subplans
  \item materialize commonly used views which result from expensive queries
  \item remove unnecessary joins
  \item reinterpret subqueries as joins
  \item pipeline where possible
  \item and more...
\end{itemize}

\end{frame}
\end{document}
