\input{configuration}

\def\ojoin{\setbox0=\hbox{$\bowtie$}%
  \rule[-.02ex]{.25em}{.4pt}\llap{\rule[\ht0]{.25em}{.4pt}}}
\def\leftouterjoin{\mathbin{\ojoin\mkern-5.8mu\bowtie}}

\title{Tutorial 8 --- Transactions and Recovery }

\author{Richard Wong \\ \small \texttt{rk2wong@edu.uwaterloo.ca}}
\institute{Department of Electrical and Computer Engineering \\
  University of Waterloo}
\date{\today}


\begin{document}

\begin{frame}
  \titlepage

\end{frame}


\begin{frame}
\frametitle{Exercise 8-1}

Is the following transaction schedule \textit{recoverable}?

Can we make a conflict-equivalent schedule that is \textit{cascadeless}?

\begin{center}
\begin{tabular}{ c c c c c c c c c c }
  \hline
  T1 & $r_x$ & & $r_y$ & & & & & & c \\
  \hline
  T2 & & $w_y$ & & & $r_x$ & & & c & \\
  \hline
  T3 & & & & $w_x$ & & $r_x$ & c & & \\
  \hline
\end{tabular}
\end{center}

\end{frame}


\begin{frame}
\frametitle{Exercise 8-1 Solution (1/2)}

In the context of recoverability, a transaction T \textbf{depends on} a transaction S if T reads a value that S has written to previously.

For a schedule to be recoverable, for any pair of transactions S and T, if T depends on S, then S must commit \textit{before} T does.

In the given schedule, that means:

\begin{itemize}
  \item T3 must commit before T2 does, and
  \item T2 must commit before T1 does.
\end{itemize}

\end{frame}


\begin{frame}
\frametitle{Exercise 8-1 Solution (2/2)}

For a schedule to be recoverable, for any pair of transactions S and T, if T depends on S, then S must commit \textit{before} T reads a dependent value.

The following schedule is conflict-equivalent to the original, and is also cascadeless.

\begin{center}
\begin{tabular}{ c c c c c c c c c c }
  \hline
  T1 & $r_x$ & & & & & & & $r_y$ & c \\
  \hline
  T2 & & $w_y$ & & & & $r_x$ & c & & \\
  \hline
  T3 & & & $w_x$ & $r_x$ & c & & & & \\
  \hline
\end{tabular}
\end{center}

Note: In the case of an abort, only one transaction will need to roll back and the others can run; this does not preserve equivalence to the original schedule, but can still be consistent.

\end{frame}


\begin{frame}
\frametitle{Exercise 8-2}

What is the weakest isolation guarantee available in SQL?

When would an developer want to use a weaker isolation level in their application?

\end{frame}


\begin{frame}
\frametitle{Exercise 8-2 Solution}

The weakest isolation level provided by SQL is \textbf{read-uncommitted}.

This level guarantees \textit{no dirty writes}: no writes on top of uncommitted writes by other transactions.

Weaker isolation affords a greater degree of concurrency, for a potential boost in throughput. Use if the possibly-resulting inconsistencies are irrelevant to the application, or they are otherwise worth dealing with.

\end{frame}


\begin{frame}
\frametitle{Exercise 8-3}

Some DMBSes use snapshot isolation to implement \textit{serializable}-level isolation. It works most of the time, but has failure cases.

How can snapshot isolation fail to create a serializable schedule, and what should happen when it creates a non-serializable schedule?

\end{frame}


\begin{frame}
\frametitle{Exercise 8-3 Solution}

In \textbf{snapshot isolation}, transactions operate on a snapshot of the DB that looks as it did when the transaction started.

A consistency check (for dirty writes, essentially) is performed right before an attempted commit.

Snapshot isolation can result in non-serializable schedules because no additional work is done to ensure serializability as the schedule is generated. The consistency check at the end will abort non-serializable schedules.

\end{frame}


\begin{frame}
\frametitle{Exercise 8-4}

Show that 2PL can create schedules that result in deadlock.

What can we do to \textbf{prevent} deadlock?

\end{frame}


\begin{frame}
\frametitle{Exercise 8-4 Solution (1/2)}

Recall that deadlock requires four conditions to be simultaneously true:

\begin{enumerate}
  \item mutual exclusion,
  \item hold-and-wait,
  \item no pre-emption, and
  \item circular wait.
\end{enumerate}

A 2PL schedule that results in deadlock, supposing T1 wants to read resources $a$ and $b$, and T2 wants to write to those same resources:

\begin{center}
\begin{tabular}{ c c c c c }
  \hline
  T1 & $s_a$ &      &      & $s_b$ \\
  \hline
  T2 &      & $x_b$ & $x_a$ &      \\
  \hline
\end{tabular}
\end{center}

\end{frame}


\begin{frame}
\frametitle{Exercise 8-4 Solution (2/2)}

Note that in 2PL, transactions can run into deadlock only during their growing phase.

We can prevent deadlock by having a smarter lock manager. There are several ways to prevent deadlock. The following are some things that the lock manager can do:

\begin{itemize}
  \item Use a deadlock detection algorithm (e.g. Banker's, graph algorithms) to determine whether granting a lock might lead to deadlock, and make the caller wait or abort if so. This can be expensive.
  \item Enforce a partial ordering (could be encoded with a tree) for lock acquisition to remove the possibility of circular wait.
\end{itemize}

\end{frame}


\begin{frame}
\frametitle{Exercise 8-5}

Show that 2PL can create recoverable schedules with cascading rollbacks.

What variant of 2PL creates cascadeless schedules?

\end{frame}


\begin{frame}
\frametitle{Exercise 8-5 Solution}

\begin{center}
\begin{tabular}{ c c c c c c c c c c c c c }
  \hline
  T1 & $x_a$ & $w_a$ & $u_a$ &      &      &      &      &      &      &      &      & abort \\
  \hline
  T2 &      &      &      & $s_a$ & $r_a$ & $x_a$ & $w_a$ & $u_a$ &      &      &      &      \\
  \hline
  T3 &      &      &      &      &      &      &      &      & $s_a$ & $r_a$ & $u_a$ &      \\
  \hline
\end{tabular}
\end{center}

\end{frame}


\begin{frame}
\frametitle{Exercise 8-6}

Supposing that instead handling deadlock with an avoidance/prevention strategy, we try to detect and recover. How do we recover from a deadlock?

\end{frame}


\begin{frame}
\frametitle{Exercise 8-6 Solution}

Recovering from a deadlock involves rolling back transactions until the circular wait is removed.

We choose a transaction to roll back using some heuristic that takes starvation into account, then roll back some or all of the transaction.

\end{frame}


\end{document}
