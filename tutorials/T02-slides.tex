\input{configuration}

\title{Tutorial 2 --- Data Definition, Security, Modelling Languages}

\author{Richard Wong \\ \small \texttt{rk2wong@edu.uwaterloo.ca}}
\institute{Department of Electrical and Computer Engineering \\
  University of Waterloo}
\date{\today}


\begin{document}

\begin{frame}
  \titlepage

\end{frame}


\begin{frame}
\frametitle{Exercise 2-1}

Oops! Our \texttt{user} table contains passwords in plaintext! Our table looks like:

\texttt{
  CREATE TABLE user ( \\
    id int, \\
    name varchar(100) NOT NULL, \\
    password varchar(2000) NOT NULL, \\
    PRIMARY KEY (id) \\
  ); \\
  INSERT INTO user VALUES (1, 'Alice', 'abcABC123!@\#');
}

How can we modify the \texttt{user} table to replace \texttt{password} with a non-nullable \texttt{hashedPassword} column, containing the result of \texttt{PASSWORD(password)} of each row?

\end{frame}


\begin{frame}
\frametitle{Exercise 2-2}

Suppose we have the following instance of the \texttt{user} table. Do the contents suggest any potential security vulnerabilities?

What could we do to improve it?

\begin{center}
  \begin{tabular}{||c l r||}
  \hline
  id & name & hashedPassword \\ [0.5ex]
  \hline\hline
  1 & Alice & *BEEFBEEFBEEFBEEF \\
  \hline
  2 & Bob & *43F23EBECA12AD31CBA2C1BC2 \\
  \hline
  3 & Charlie & *BEEFBEEFBEEFBEEF \\
  \hline
  4 & Donna & *43DBA275606D7A633AC28 \\
  \hline
\end{tabular}
\end{center}

\end{frame}


\begin{frame}
\frametitle{Exercise 2-3}

Which of the following functions are deterministic?

\texttt{ABS} \\
\texttt{COUNT} \\
\texttt{DATEDIFF} \\
\texttt{GETDATE} \\
\texttt{ISNULL} \\
\texttt{RAND} \\

\end{frame}


\begin{frame}
\frametitle{Exercise 2-4}

What could the database schema for the following E-R diagram look like?

\begin{center}
  \includegraphics[width=0.8\textwidth]{images/weak-entity-set}
\end{center}

\end{frame}


\begin{frame}
\frametitle{Exercise 2-5}

What do the following parts of this diagram mean?

\begin{itemize}
  \item the solid underline on \textit{course\_id}
  \item the single arrow pointing to \textit{course}
  \item the double diamond around \textit{sec\_course}
  \item the double lines between \textit{sec\_course} and \textit{section}
  \item the dashed underlines on the attributes of \textit{section}
\end{itemize}

\begin{center}
  \includegraphics[width=0.8\textwidth]{images/weak-entity-set}
\end{center}

\end{frame}


\begin{frame}
\frametitle{Exercise 2-6}

What could the database schema for the following E-R diagram look like?

\begin{center}
  \includegraphics[width=0.5\textwidth]{images/specialization-generalization}
\end{center}

\end{frame}


\begin{frame}
\frametitle{Fin}

What other topics do you want to talk about?

\end{frame}
\end{document}
