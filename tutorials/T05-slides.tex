\input{configuration}

\title{Tutorial 5 --- Query Optimization, Planning, Evaluation }

\author{Richard Wong \\ \small \texttt{rk2wong@edu.uwaterloo.ca}}
\institute{Department of Electrical and Computer Engineering \\
  University of Waterloo}
\date{\today}


\begin{document}

\begin{frame}
  \titlepage

\end{frame}


\begin{frame}
\frametitle{Exercise 5-1}

\begin{enumerate}
  \item What are the two metrics we will use to estimate query operation costs?
  \item What does each metric represent?
  \item How do we use the metrics to arrive at an estimate?
\end{enumerate}

\end{frame}


\begin{frame}
\frametitle{Exercise 5-2}

Suppose we run a query that performs a single-attribute GREATER THAN comparison in its WHERE clause.

e.g. \\
SELECT * FROM people \\
WHERE age > 20

How might the following evaluation strategies impact the cost of the operation?

\begin{enumerate}
  \item Use a primary index if there is one.
  \item Use a secondary index if there is one.
  \item Use a linear scan.
\end{enumerate}

What if we were performing a LESS THAN comparison?

\end{frame}


\begin{frame}
\frametitle{Exercise 5-3}

Derive the worst-case and best-case cost estimate for a block nested-loop join.

\end{frame}


\begin{frame}
\frametitle{Exercise 5-4}

Derive the formulas for \textit{selectivity} of the following types of selections:

Recall that selectivity is the estimated probability that a tuple matches a selection criterion.

\begin{enumerate}
  \item conjunction (WHERE $\theta_1$ AND $\theta_2$ AND ... AND $\theta_m$)
  \item negation (WHERE NOT $\theta$)
  \item disjunction (WHERE $\theta_1$ OR $\theta_2$ OR ... OR $\theta_m$)
\end{enumerate}

\end{frame}


\begin{frame}
\frametitle{Exercise 5-5}

What are some good rules of thumb that a query optimizer could use to reduce the cost of query plan selection, or the cost of the query itself?

\end{frame}


\end{document}
