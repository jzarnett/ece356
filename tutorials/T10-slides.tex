\input{configuration}

\def\ojoin{\setbox0=\hbox{$\bowtie$}%
  \rule[-.02ex]{.25em}{.4pt}\llap{\rule[\ht0]{.25em}{.4pt}}}
\def\leftouterjoin{\mathbin{\ojoin\mkern-5.8mu\bowtie}}

\title{Tutorial 10 --- Parallelism and Distributed Databases }

\author{Richard Wong \\ \small \texttt{rk2wong@edu.uwaterloo.ca}}
\institute{Department of Electrical and Computer Engineering \\
  University of Waterloo}
\date{\today}


\begin{document}

\begin{frame}
  \titlepage

\end{frame}


\begin{frame}
\frametitle{Exercise 10-1}

What kinds of queries are the following partitioning schemes well-suited for?

\begin{enumerate}
  \item round-robin
  \item range partitioning
  \item hash partitioning
\end{enumerate}

\end{frame}


\begin{frame}
\frametitle{Exercise 10-2}

How would a distributed DB using the following partitioning schemes handle \textbf{addition} of nodes?

\begin{enumerate}
  \item round-robin
  \item range partitioning
  \item hash partitioning
\end{enumerate}

\end{frame}


\begin{frame}
\frametitle{Exercise 10-3}

What factors would account for \textbf{load imbalance} in the following partitioning schemes?

\begin{enumerate}
  \item range partitioning
  \item hash partitioning
\end{enumerate}

\end{frame}


\begin{frame}
\frametitle{Exercise 10-4}

How can we alleviate the problem of load imbalance in range partitioning?

\end{frame}


\begin{frame}
\frametitle{Exercise 10-5}

Suppose we have the following relation: \\
\begin{center}
  \texttt{employee(name, address, salary, plantNumber)}
\end{center}

The relation is \textbf{fragmented horizontally}, \\
and each fragment has a \textbf{local replica}, \\
and a \textbf{replica in New York}.

Provide a reasonable processing strategy for each of the following queries \textbf{made from the plant in Montreal}:

\begin{enumerate}
  \item Find all employees at the Toronto plant.
  \item Find the average salary of all employees.
  \item Find the highest-paid employee in Toronto, Vancouver, and Edmonton.
\end{enumerate}

\end{frame}

\end{document}
