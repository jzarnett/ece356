\input{configuration}

\title{Lecture --- Relational Query Languages \& SQL, Continued}

\author{Jeff Zarnett \\ \small \texttt{jzarnett@uwaterloo.ca}}
\institute{Department of Electrical and Computer Engineering \\
  University of Waterloo}
\date{\today}


\begin{document}

\begin{frame}
  \titlepage

 \end{frame}



\begin{frame}
\frametitle{Set Difference}

The \alert{set difference} operation works more or less as you would expect: it takes a relation and removes from it any tuples that are in the second relation. 

\begin{center}
	\includegraphics[width=0.5\textwidth]{images/takeheraway.jpg}
\end{center}

After taking away things we do not want, we are left with the things we do want. 

Its input is therefore two relations and it produces a new relation as its output.


\end{frame}



\begin{frame}
\frametitle{Set Difference}

The mathematical symbol is $-$ (the minus or subtraction operator). 

The notation is then $r_{1} - r_{2}$ where $r_{1}$ and $r_{2}$ are relations.

 This produces a relation $r_{3}$ that contains all tuples in $r_{1}$ that are not in $r_{2}$. 
 
 It is possible to chain the subtraction operators, but like regular mathematical subtraction, it does not commute: the order matters a lot. 


\end{frame}




\begin{frame}
\frametitle{Set Difference}

The query $\sigma_{make = ''Honda''}( vehicle ) -  \sigma_{year < ''2010''}( vehicle )$ produces the results: 

{\small
\begin{center}
\begin{tabular}{|l|l|l|l|l|} \hline
	\textbf{VIN} & \textbf{year} & \textbf{make} & \textbf{model} & \textbf{license\_plate\_number} \\ \hline
	1GYS3BKJ5FR338462 & 2016 & Honda & Civic & YYYY 585 \\ \hline	
\end{tabular}
\end{center}
}

\end{frame}



\begin{frame}
\frametitle{Set Difference}

Just like the union operation, the sets must be compatible. The same rules from union apply: 

\begin{enumerate}
	\item The relations must have the same number of attributes.
	\item The domain of attribute $i$ in the first relation must be the same as the domain of attribute $i$ in the second relation, for all $i$. 
\end{enumerate}

\end{frame}



\begin{frame}
\frametitle{Set Difference}

In SQL the keyword we need for this is \texttt{EXCEPT}\footnote{Although Oracle uses \texttt{MINUS} instead.}. 

Like the union operation, parenthesis prevent confusion and/or make your database server less sad.

\texttt{(SELECT make, model FROM vehicle WHERE make = 'Volkswagen')\\
EXCEPT\\
(SELECT make, model, FROM vehicle WHERE year < 2016);}

Produces as output: 

\begin{center}
\begin{tabular}{|l|l|} \hline
\textbf{make} & \textbf{model} \\ \hline
	Volkswagen & Jetta  \\ \hline
\end{tabular}
\end{center}

\end{frame}



\begin{frame}
\frametitle{Alternative Route}

Those with experience in databases may also think about \texttt{NOT IN}. 

The \texttt{EXCEPT} keyword encompasses both the \texttt{DISTINCT} and \texttt{NOT IN} behaviour, so duplicates are eliminated. 

Example:

\texttt{SELECT make, model FROM vehicle WHERE make = 'Volkswagen' AND 
VIN NOT IN\\ (SELECT VIN FROM vehicle WHERE year < 2016);}

This demonstrates a subquery: the first query selects a relation with one attribute.  

\end{frame}



\begin{frame}
\frametitle{Cartesian Product}

The \alert{cartesian product} combines information from two relations. 

It is often the case that the data we need is in more than one relation. 

That might be a design problem, but good design means data is in there only once and sometimes we do need to combine relations. 

Suppose we wanted to send reminder notices to people whose license plates will expire next month. We combine license plate data with address data. 

\end{frame}



\begin{frame}
\frametitle{Cartesian Product}


The mathematical symbol for cartesian product is $\times$, the multiplication symbol. 

The cartesian product $r_{1} \times r_{2}$ forms a third relation $r_{3}$.

 The new relation has all the attributes of both the relations that went into it (in the order specified). 

\end{frame}



\begin{frame}
\frametitle{Cartesian Product Example}

Let us pretend our license plate relation contains exactly two tuples:

{\small
\begin{center}
\begin{tabular}{|l|l|l|}\hline
	\textbf{number} & \textbf{expiry} & \textbf{owner\_address\_id} \\ \hline
	ZZZZ 123 & 2018-09-30 & 24601 \\ \hline
	AAAA 855 & 2019-04-01 & 12949 \\ \hline
\end{tabular}
\end{center}
}

And our owner address set is also two tuples: 

{\small
\begin{center}
	\begin{tabular}{|l|l|l|l|l|l|}\hline
		\textbf{id} & \textbf{name} &\textbf{street} & \textbf{city} & \textbf{province} & \textbf{postal\_code} \\ \hline
		24601 & Jean Valjean & 19 Rue des Prisonniers & Ottawa & ON & B1B 1B1\\ \hline
		12949 & Alice Jones & 4 Generic Place & Kenora & ON & C2C 2C2\\ \hline
	\end{tabular}
\end{center}
}


\end{frame}



\begin{frame}
\frametitle{Cartesian Product Example}

Then the query $license\_plate \times owner\_address$ produces:

{\tiny
\begin{center}
	\begin{tabular}{|l|l|l|l|l|l|l|l|l|}\hline
		\textbf{number} & \textbf{expiry} & \textbf{owner\_address\_id} & \textbf{id} & \textbf{name} &\textbf{street} & \textbf{city} & \textbf{province} & \textbf{postal\_code} \\ \hline
		ZZZZ 123 & 2018-09-30 & 24601 & 24601 & Jean Valjean & 19 Rue des Prisonniers & Ottawa & ON & B1B 1B1\\ \hline
ZZZZ 123 & 2018-09-30 & 24601 & 12949 & Alice Jones & 4 Generic Place & Kenora & ON & C2C 2C2\\ \hline
		AAAA 855 & 2019-04-01 & 12949 & 24601 & Jean Valjean & 19 Rue des Prisonniers & Ottawa & ON & B1B 1B1\\ \hline
		AAAA 855 & 2019-04-01 & 12949 & 12949 & Alice Jones & 4 Generic Place & Kenora & ON & C2C 2C2\\ \hline
	\end{tabular}
\end{center}
}

This is all possibilities!

\begin{center}
	\includegraphics[width=0.3\textwidth]{images/visibleconfusion.jpg}
\end{center}

\end{frame}



\begin{frame}
\frametitle{Cartesian Product's Cardinal Problem}

Each tuple from the first relation is paired with each tuple from the second. 

So if $r_{1}$ contains $x$ tuples and $r_{2}$ contains $y$ tuples, there will be $x*y$ tuples in the resulting relation... 

The vast majority of which are garbage! 

\end{frame}



\begin{frame}
\frametitle{Duplicate Attribute Names}

If we had two names that were the same, we need a way to differentiate them. 

Suppose that two relations, one called \texttt{book} and one called \texttt{author}, each having an attribute \texttt{id}. 

Prefix the attribute with the name of the relation from which it came. 

Thus, in the cartesian product the two columns would be \texttt{book.id} and \texttt{reader.id}.


\end{frame}



\begin{frame}
\frametitle{Cartesian Product in SQL}

In SQL it is very simple to get the cartesian product:\\
\texttt{SELECT * FROM license\_plate, owner\_address;} 

Separate the relations we wish to appear in the cartesian product with a comma.

We can always combine more if we needed.

\end{frame}



\begin{frame}
\frametitle{Makes No Sense}

We need to restrict our query with a selection predicate. 

If the two relations are connected by some ID in in common, we need that. 

In this case, the \texttt{id} attribute in the address matches \texttt{owner\_address\_id} in the license plate relation. 

So $\sigma_{owner\_address\_id = id}( license\_plate \times owner\_address )$.

\end{frame}

\begin{frame}
\frametitle{Makes No Sense}

Or in SQL: \texttt{SELECT * FROM license\_plate, owner\_address where owner\_address\_id = id;} 

This may seem unclear so we might prefer to prefix these with their names: \texttt{SELECT * FROM license\_plate, owner\_address where license\_plate.owner\_address\_id = owner\_address.id;} 


\end{frame}



\begin{frame}
\frametitle{Adding the Restriction to Cartesian Product}


{\tiny
\begin{center}
	\begin{tabular}{|l|l|l|l|l|l|l|l|l|}\hline
		\textbf{number} & \textbf{expiry} & \textbf{owner\_address\_id} & \textbf{id} & \textbf{name} &\textbf{street} & \textbf{city} & \textbf{province} & \textbf{postal\_code} \\ \hline
		ZZZZ 123 & 2018-09-30 & 24601 & 24601 & Jean Valjean & 19 Rue des Prisonniers & Ottawa & ON & B1B 1B1\\ \hline
		AAAA 855 & 2019-04-01 & 12949 & 12949 & Alice Jones & 4 Generic Place & Kenora & ON & C2C 2C2\\ \hline
	\end{tabular}
\end{center}
}


\end{frame}



\begin{frame}
\frametitle{Rename}

The \alert{rename} operation is used to both change the name of existing attributes/relations and assign names to ones that have no name. 

The mathematical notation is $\rho$ (rho).

\begin{center}
	\includegraphics[width=0.7\textwidth]{images/asvader.jpg}
\end{center}

\end{frame}



\begin{frame}
\frametitle{Rename}

 If applied to a relation, it is $\rho_{x}(E)$ where it renames the relation to $x$. 
 
Or it can be applied to the name of the attributes: $\rho_{x(a_{1}, a_{2}, a_{3}, a_{4}...)}(E)$ which renames the relation to $x$ and then the attributes $a_{1}$...

\end{frame}



\begin{frame}
\frametitle{For Dinner, or AS Dinner?}

In SQL we can use the \texttt{AS} keyword. 

That is not super interesting on a simple query to rename a relation:\\
 \texttt{SELECT * FROM vehicle AS autos;} renames the relation result to ``autos''.
 
 It makes some more sense if we are going to use it in a subquery it will change the prefix for any duplicate attribute names.
 
 Example: change \texttt{book.id} to \texttt{textbook.id}.



\end{frame}


\begin{frame}
\frametitle{Further Use of AS}

A more likely use is to rename the attributes of a relation.

The motivation for the rename operation right now probably seems weak. 

There are some situations where we will need to use the rename operation.

\end{frame}




\begin{frame}
\frametitle{Set Intersection}

\alert{Set intersection} is exactly what it sounds like based on our understanding of mathematical sets. 

The symbol for it is $\cap$ and it is equivalent to $r_{1} - (r_{1} - r_{2})$.

\end{frame}



\begin{frame}
\frametitle{Set Intersection in SQL}

In SQL they keyword is \texttt{INTERSECT}. 

Its use is limited, however, in a way similar to union. 

If we want to use intersection on two queries of the same relation, we could just as easily set it up as a selection with a compound predicate.

\texttt{(SELECT name, street, city, province, postal\_code FROM owner\_address)\\ 
INTERSECT\\ 
(SELECT name, street, city, province, postal\_code FROM employee);}


\end{frame}




\begin{frame}
\frametitle{Assignment}

The \alert{assignment} operation allows us to take the result of an expression and put it in a temporary variable. 

The mathematical symbol for this is $\leftarrow$.

This is just notational convenience for the benefit of the reader. 

\begin{center}
	\includegraphics[width=0.4\textwidth]{images/writedown.jpg}
\end{center}

\end{frame}




\begin{frame}
\frametitle{Assignment}

So instead of something $\sigma_{owner\_address\_id = id}( license\_plate \times owner\_address )$, we could write: 

$temp \leftarrow license\_plate \times owner\_address$


$\sigma_{owner\_address\_id = id}( temp ) $.



\end{frame}



\begin{frame}
\frametitle{Assignment in SQL}

SQL does have an assignment operator: \texttt{:=}. 

I will, however, discourage you from using it.

The reason I discourage the use of assignment is the need to get away from the C-like thinking with counters and iteration.


\end{frame}




\begin{frame}
\frametitle{Join}

The \alert{Join} operations combines cartesian product with a selection.

The mathematical symbol for the \textit{natural join} is $\Join$ (bowtie). 

\begin{center}
	\includegraphics[width=0.4\textwidth]{images/bowties.jpg}
\end{center}

\end{frame}




\begin{frame}
\frametitle{Join}

It combines the cartesian product with a selection, forcing equality. 

A natural join combined with a selection is a theta join $\Join_{\theta}$ where $\theta$ is the selection predicate.

 Note that it is associative: so we can chain it as we need: 
 
 $vehicle \Join license\_plate \Join owner\_address$ is equivalent to  $(vehicle \Join license\_plate) \Join owner\_address$  and $vehicle \Join (license\_plate \Join owner\_address)$.


\end{frame}



\begin{frame}
\frametitle{Same Name?}

Except, this probably doesn't work the way we hope on our sample data because the fields are not named the same thing. 

If both fields are called \texttt{id}, the database server will try to match on those. 

But this is not going to work for our data since they all have different names. 


\end{frame}



\begin{frame}
\frametitle{Join in SQL}

In SQL the keyword we need for this is \texttt{NATURAL JOIN}. 

So we would do: \texttt{SELECT * FROM vehicle NATURAL JOIN license\_plate;} 

Again, though, this is not going to cooperate for us because our names don't match.

What do we do then?

\end{frame}



\begin{frame}
\frametitle{Inner Join}

The default type is the \texttt{INNER JOIN} and it's what you get if you just write \texttt{JOIN} instead of being specific. 

It requires us to use the \texttt{ON} clause to specify how we relate one side to the other.

\texttt{SELECT * FROM license\_plate JOIN owner\_address ON owner\_address\_id = id;} produces:

{\tiny
\begin{center}
	\begin{tabular}{|l|l|l|l|l|l|l|l|l|}\hline
		\textbf{number} & \textbf{expiry} & \textbf{owner\_address\_id} & \textbf{id} & \textbf{name} &\textbf{street} & \textbf{city} & \textbf{province} & \textbf{postal\_code} \\ \hline
		ZZZZ 123 & 2018-09-30 & 24601 & 24601 & Jean Valjean & 19 Rue des Prisonniers & Ottawa & ON & B1B 1B1\\ \hline
		AAAA 855 & 2019-04-01 & 12949 & 12949 & Alice Jones & 4 Generic Place & Kenora & ON & C2C 2C2\\ \hline
	\end{tabular}
\end{center}
}

\end{frame}



\begin{frame}
\frametitle{Use ON}

Use the \texttt{ON} clause rather than \texttt{NATURAL JOIN}. 

Oftentimes you need it and when you don't you sometimes get the wrong behaviour because a natural join does the wrong thing. 

If the license plate relation called the plate number ``id'' instead of ``number'', the natural join operation would run but produce no results. 

Be explicit about how the join should work.


\end{frame}



\begin{frame}
\frametitle{Outer Join?}

If there is an inner join, there is obviously an outer join. Actually, three!

Suppose we have a license plate that does not correspond to any address: 

{\scriptsize
\begin{center}
\begin{tabular}{|l|l|l|}\hline
	\textbf{number} & \textbf{expiry} & \textbf{owner\_address\_id} \\ \hline
	ZZZZ 123 & 2018-09-30 & 24601 \\ \hline
	AAAA 855 & 2019-04-01 & 12949 \\ \hline
	BBCC 394 & 2019-02-12 & null \\ \hline
\end{tabular}
\end{center}
}

and an address that does not correspond to any license plate:

{\scriptsize
\begin{center}
	\begin{tabular}{|l|l|l|l|l|l|}\hline
		\textbf{id} & \textbf{name} &\textbf{street} & \textbf{city} & \textbf{province} & \textbf{postal\_code} \\ \hline
		24601 & Jean Valjean & 19 Rue des Prisonniers & Ottawa & ON & B1B 1B1\\ \hline
		12949 & Alice Jones & 4 Generic Place & Kenora & ON & C2C 2C2\\ \hline
		86753 & Sherlock Holmes & 221B Baker St & London & ON & D4D 4D4\\ \hline
	\end{tabular}
\end{center}
}

\end{frame}


\begin{frame}
\frametitle{Inner Join vs Outer Join}

The inner join of license plate with owner address would look like this: 

{\tiny
\begin{center}
	\begin{tabular}{|l|l|l|l|l|l|l|l|l|}\hline
		\textbf{number} & \textbf{expiry} & \textbf{owner\_address\_id} & \textbf{id} & \textbf{name} &\textbf{street} & \textbf{city} & \textbf{province} & \textbf{postal\_code} \\ \hline
		ZZZZ 123 & 2018-09-30 & 24601 & 24601 & Jean Valjean & 19 Rue des Prisonniers & Ottawa & ON & B1B 1B1\\ \hline
		AAAA 855 & 2019-04-01 & 12949 & 12949 & Alice Jones & 4 Generic Place & Kenora & ON & C2C 2C2\\ \hline
	\end{tabular}
\end{center}
}


\end{frame}



\begin{frame}
\frametitle{Left Outer Join}

If we use the left outer join, which has the symbol {\tiny \textifsym{d|><|}}? 

We get a tuple in the result for each entry in the left (first) table, even if it has no match in the second. 

For such a tuple, all attributes from the right (second) table will simply be ``null''. 

If a tuple in the right table has no match in the left table, it doesn't appear at all.


\end{frame}



\begin{frame}
\frametitle{Left Outer Join}

The SQL for this is, of course, \texttt{LEFT OUTER JOIN}. So \texttt{SELECT * FROM license\_plate LEFT OUTER JOIN owner\_address ON owner\_address\_id = id;} returns:

{\tiny
\begin{center}
	\begin{tabular}{|l|l|l|l|l|l|l|l|l|}\hline
		\textbf{number} & \textbf{expiry} & \textbf{owner\_address\_id} & \textbf{id} & \textbf{name} &\textbf{street} & \textbf{city} & \textbf{province} & \textbf{postal\_code} \\ \hline
		ZZZZ 123 & 2018-09-30 & 24601 & 24601 & Jean Valjean & 19 Rue des Prisonniers & Ottawa & ON & B1B 1B1\\ \hline
		AAAA 855 & 2019-04-01 & 12949 & 12949 & Alice Jones & 4 Generic Place & Kenora & ON & C2C 2C2\\ \hline
		BBCC 394 & 2019-02-12 & null & null & null & null & null & null & null \\ \hline
	\end{tabular}
\end{center}
}


\end{frame}



\begin{frame}
\frametitle{Right Outer Join}


The right outer join, symbol, {\tiny \textifsym{|><|d}} (\texttt{RIGHT OUTER JOIN}) is the mirror image operation of the left outer join. 

We get a tuple in the result for each entry in the right (second) table, even if it has no match in the first. 

For such a tuple, all attributes from the left (first) table will simply be ``null''. 

If a tuple in the left table has no match in the right table, it doesn't appear at all.


\end{frame}




\begin{frame}
\frametitle{Right Outer Join}


\texttt{SELECT * FROM license\_plate RIGHT OUTER JOIN owner\_address ON owner\_address\_id = id;} returns:

{\tiny
\begin{center}
	\begin{tabular}{|l|l|l|l|l|l|l|l|l|}\hline
		\textbf{number} & \textbf{expiry} & \textbf{owner\_address\_id} & \textbf{id} & \textbf{name} &\textbf{street} & \textbf{city} & \textbf{province} & \textbf{postal\_code} \\ \hline
		ZZZZ 123 & 2018-09-30 & 24601 & 24601 & Jean Valjean & 19 Rue des Prisonniers & Ottawa & ON & B1B 1B1\\ \hline
		AAAA 855 & 2019-04-01 & 12949 & 12949 & Alice Jones & 4 Generic Place & Kenora & ON & C2C 2C2\\ \hline
		null & null & null & 86753 & Sherlock Holmes & 221B Baker St & London & ON & D4D 4D4 \\ \hline
	\end{tabular}
\end{center}
}

\end{frame}



\begin{frame}
\frametitle{Full Outer Join}

Finally, there is the full outer join, with the symbol {\tiny \textifsym{d|><|d}}.

It does both the left and right outer join operations, including tuples from each relation that do not match the other: 

{\tiny
\begin{center}
	\begin{tabular}{|l|l|l|l|l|l|l|l|l|}\hline
		\textbf{number} & \textbf{expiry} & \textbf{owner\_address\_id} & \textbf{id} & \textbf{name} &\textbf{street} & \textbf{city} & \textbf{province} & \textbf{postal\_code} \\ \hline
		ZZZZ 123 & 2018-09-30 & 24601 & 24601 & Jean Valjean & 19 Rue des Prisonniers & Ottawa & ON & B1B 1B1\\ \hline
		AAAA 855 & 2019-04-01 & 12949 & 12949 & Alice Jones & 4 Generic Place & Kenora & ON & C2C 2C2\\ \hline
		BBCC 394 & 2019-02-12 & null & null & null & null & null & null & null \\ \hline
		null & null & null & 86753 & Sherlock Holmes & 221B Baker St & London & ON & D4D 4D4 \\ \hline
	\end{tabular}
\end{center}
}


\end{frame}





\end{document}

